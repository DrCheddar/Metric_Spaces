\documentclass{article}
\usepackage[english]{babel}
\usepackage[utf8]{inputenc}
\usepackage{fancyhdr}
\usepackage{array}
\usepackage{graphicx,amsmath,amsfonts,amssymb, epsfig,color,url, amsthm}
\usepackage{siunitx}
\usepackage{setspace}
\usepackage{subcaption}
\usepackage{makecell}
\usepackage{url}
\usepackage{longtable}
\usepackage{csvsimple}
\usepackage{placeins} % put this in your pre-amble
\usepackage{flafter}
\usepackage{mathrsfs} % https://www.ctan.org/pkg/mathrsfs
\usepackage[version=4]{mhchem}
\usepackage{bm}
\usepackage{natbib}
\usepackage{mathtools}
\usepackage{braket}
\usepackage[shortlabels]{enumitem}

\sisetup{detect-all}

\makeatletter
\providecommand\add@text{}
\newcommand\tagaddtext[1]{%
  \gdef\add@text{#1\gdef\add@text{}}}% 
\renewcommand\tagform@[1]{%  
  \maketag@@@{\llap{\add@text\quad}(\ignorespaces#1\unskip\@@italiccorr)}%
}
\newcommand{\unit}[1]{\ensuremath{\, \mathrm{#1}}}
\newcommand{\B}[1]{\boldsymbol{#1}}

\newcommand*\R{\mathbb{R}}
\newcommand*\functionSpaceG{\ensuremath{\mathscr{F}(\mathcal{D})}}
\newcommand*\funcSpace{\mathscr{F}_M(\mathcal{D})}
\newcommand*\basisSet{\{\B{u}_n\}}


% \newcommand*\r{\rangle}
% \newcommand*\l{\langle}
\newcommand{\refeqn}[1]{(\refeq{#1})}

\newcommand*\bigcdot{\mathpalette\bigcdot@{.5}}
\newcommand*\bigcdot@[2]{\mathbin{\vcenter{\hbox{\scalebox{#2}{$\m@th#1\bullet$}}}}}

\DeclareRobustCommand\iff{\;\Longleftrightarrow\;}

\DeclarePairedDelimiter\abs{\lvert}{\rvert}%
\DeclarePairedDelimiter\norm{\lVert}{\rVert}%

\pagestyle{fancy}
\fancyhf{}
\lhead{Finding The Best Approximation Of A Function Using A Basis Set}
\rfoot{Page \thepage}
\linespread{1}
\title{
  {\LARGE
   \textbf{Finding The Best Approximation Of A Function Using A Basis Set}
   }\\~\\
   {\large
    Treating Functions As A Vector, How Can A Representation Be Constructed Using A Set Of Basis Functions?
   }\\~\\
  {\large Mathematics Internal Assessment}\\
  % {\large Colonel By Secondary School}
}
% \author{Jeffrey Li} 
\date{\today}


\newtheorem{theorem}{Theorem}[section] 
\newtheorem{lemma}[theorem]{Lemma}
\theoremstyle{definition}
\newtheorem{definition}{Definition}[section]
\newtheorem{corollary}{Corollary}[definition]
\newtheorem{proposition}{Proposition}

\newtheorem{post}{Postulate}[section]

\theoremstyle{remark}
\newtheorem*{remark}{Remark}

\doublespacing

\begin{document}

\maketitle
\pagebreak


\tableofcontents
\pagebreak

\section{Research Question}
Treating Functions As A Vector, How Can A Representation Be Constructed Using A Set Of Basis Functions?

\section{Introduction}
For me, the idea of finding sets of basis functions that can be combined to approximate 
others, first came to me when we were learning about the 
Taylor Series representation of differentiable functions in class.

Using the Taylor Series, any infinitely differentiable 
function can be written as the sum of a polynomial series,
using n-th degree derivatives as as coefficients on the 
respective power of x.

Afterwards, when learning linear algebra, I realized that 
functions shared many of the same crucial properties with vectors.
In this light, the Taylor Series expansion can be seen 
as expressing the original function in terms of a set 
of basis functions, just like how any three dimensional 
vector can be written as a linear combination of the $\hat{i}, \hat{j}$ and $\hat{k}$
basis vectors.

My goal in this investigation therefore became to determine
what conditions were allowed for a function to be 
approximated using a linear combination of some other set of basis functions,
using properties of vectors to analyze this problem.

\section{Vector Spaces}

Before any properties of functions as vectors can be discussed, 
a formal definition of a vector space must first be defined. 
From \cite{AdvancedCalculus}, a vector space is defined the following
manner: 


\begin{theorem}
  The set of lefthand continuous functions $\mathscr{F}_{\mathcal{L}}$ is the
  maximal subset of $\mathscr{F}$ that satisfies the properties of an IPS.
\end{theorem}



\begin{theorem}
  Assume exists vector space $\mathbb{V}$ and IPS $\braket{|}$.
  For any two linearly independent elements $\vec{v}, \vec{w} \in \mathbb{V}, \vec{v}, \vec{w} \neq \vec{0}$, there exists
  inner product $\braket{|}^\prime$ on $\mathbb{V}$ such that $\braket{\vec{v}|\vec{w}}^\prime = 0$.
  
  Define $\vec{w}^\prime$ to be the element that $\vec{v} + \vec{w}^\prime \triangleq \vec{w}$,
  and $\braket{\vec{v}|\vec{w}^\prime} \triangleq \vec{0}$. For any element 
  $\vec{p} = p_v \vec{v} + p_w \vec{w} + \vec{p}_{\perp}$ where $\vec{p}_\perp$ 
  is any element orthogonal to both $\vec{v}$ and $\vec{w}$.

    
  Define 
  \begin{equation}
    \braket{\vec{p}~|~\vec{q}}^\prime \triangleq p_v q_v + p_w q_w + \braket{\vec{p}_{\perp}| \vec{q}_{\perp}}
  \end{equation}

  Then $\braket{|}^\prime$ is a valid inner product on $\mathbb{V}$.

\end{theorem}
\begin{proof}
  
\end{proof}
\begin{definition}[Vector Space]
  A Vector Space $\mathbb{V}$ is defined along with a scalar field $\mathbb{S}$
  to be any set that has the following properties:

  \begin{enumerate}
    \item Scalar Multiplication Operator: $\B{x} \in \mathbb{V}, c \in \mathbb{S} \rightarrow c \B{x} \in \mathbb{V}$
    \item Commutative Vector Addition Operator: $\B{x}, \B{y} \in \mathbb{V} \rightarrow \B{x} + \B{y} \in \mathbb{V}$
    \item Scalar Multiplication is Distributive Across Vectors: $c (\B{x} + \B{y}) = c \B{x} + c \B{y}$
    \item Scalar Multiplication is Distributive Across Scalars: $ (c_1 + c_2) \B{x} = c_1 \B{x} + c_2 \B{x}$
    \item $\vec{0}$ Vector Element: $\B{x} + \vec{0} = \B{x} ~ \forall \B{x} \in \mathbb{V}$
    \item 0 Scalar Element: $0 \B{x} = \vec{0} ~ \forall \B{x} \in \mathbb{V}$
    \item Identity Scalar Element: $1 \B{x} = \B{x} ~ \forall \B{x} \in \mathbb{V}$ 
  \end{enumerate}
  \label{def:vectorSpace}
\end{definition}

From this, we can rigorously show that the set of functions
on $x$ do indeed satisfy the conditions of a vector space - more 
specifically the set of functions defined over some common domain.

\begin{theorem}[Function Space]
  Let $\mathcal{D} \subset \R$ be any domain over the real numbers. 
  Let $\mathscr{F}(\mathcal{D})$ be defined as the set of all functions $f(x) : \mathbb{R} \to \mathbb{R}$ well defined over $\mathcal{D}$. 
  In conjunction with the scalar field $\R$, this set defines a vector space satisfying definition (\ref{def:vectorSpace}).
\end{theorem}

\begin{proof}
  Let $f(x)$ be any element in the set $\mathscr{F}(\mathcal{D})$. For any scalar $c$, $cf(x)$ is also a well defined function 
  defined over $\mathcal{D}$ and is therefore also in $\mathscr{F}(\mathcal{D})$, satisfying condition 1. Additionally,
  the sum of two functions $f(x) + g(x)$ is well defined over $\mathcal{D}$, satisfying condition 2. 
  Conditions 3 and 4  are defined as part of the algebra of functions.
  The $\vec{0}$ function element can be defined to be $\vec{0}(x) = 0$,
  so that $f(x) + \vec{0}(x) = f(x)$, satisfying condition 5. Finally,
  conditions 6 and 7 are satisfied with the conventional scalars of 0 and 1,
  both of which, when multiplied into any function, produce results in correspondance
  to the definition, showing that the set $\mathscr{F}(\mathcal{D})$ is indeed a valid vector space.

\end{proof}

\section{Defining An Inner Product}
One of the most powerful things about vector spaces is they allow any member element to 
be written as a linear combination of a much smaller set of standard basis vectors.

However, generally to determine these coefficients, an inner product is needed.
In a normal 2 or 3 dimensional vector space, the inner product, or the dot product,
can give a measure of the degree to which two vectors are geometrically `close',
determined by the angle between the two vectors. The Inner Product doesn't need
to be restricted to this geometric definition though, and can be generalized
to allow for a measurement of `closeness' encompassing vector spaces in a broader sense.

\begin{definition}[Inner Product]
  Given a vector space $\mathbb{V}$ with a corresponding scalar field $\mathbb{S}$, it is said to posses and inner product if there exists
  a function $\braket{\B{x}| \B{y}} : \mathbb{V} \times \mathbb{V} \to \mathbb{S}$ defined over the entire vector space that satisfies the following conditions:

  \begin{enumerate}
    \item Scalar Associative in Second Argument: $\braket{\B{x}|c\B{y}} = c\braket{\B{x}| \B{y}} ~ \forall \B{x}, \B{y} \in \mathbb{V}, c \in \mathbb{S}$
    \item Distributive In Second Argument: $\braket{\B{x}| ~ \B{y} + \B{z}} = \braket{\B{x}| \B{y}} + \braket{\B{x}|\B{z}}  ~ \forall \B{x}, \B{y}, \B{z} \in \mathbb{V}$
    \item Symmetry: $\braket{\B{x}| \B{y}} = \braket{\B{y}|\B{x}} ~ \forall \B{x}, \B{y} \in \mathbb{V}$
    \item Positive Semi-Definite: $\braket{\B{x}| \B{x}} > 0 \leftrightarrow \B{x} \neq \vec{0}, \braket{\B{x}| \B{x}} = 0 \leftrightarrow \B{x} = \vec{0}$
  \end{enumerate}
  \label{def:InnerProduct}
\end{definition}

\subsection{Finding A Valid Inner Product in $\mathscr{F}$}

In general, an inner product can be any mapping of two vectors into a scalar that satisfies the conditions outline above.
However, taking into account the value of the function over its entire 
domain and mapping that into a scalar is most commonly associated with 
taking the definite integral of the function. As such, it is especially useful
to define the inner product of two functions as follows, noting that
this definition satisfies all the properties previous outlined on the 
inner product. 

\begin{theorem}[Inner Product In Function Space]
  Let the domain $\mathcal{D}$ be equal to some bounded interval $[a, b] \in \R$.
  Let $\mathscr{F}_M(\mathcal{D})$ be the subset of $\mathscr{F}(\mathcal{D})$ containing every 
  measurable function over the domain, or for which 
  $$ \int_a^b f(x) dx $$ is defined. Then define the function 
  \begin{equation}
    \braket{f(x)| g(x)} = \int_a^b f(x) g(x) dx
    \label{eq:standardIP}
  \end{equation}
  This function is defined and constitutes a valid inner product for all $f(x), g(x) \in \mathscr{F}_M(\mathcal{D})$.
\end{theorem}

% \begin{proof}
%   Firstly, note that 
% \end{proof}

\section{Orthogonal Functions}

Now that an inner product has been established, we still need to 
outline some properties on the set of basis vectors we are using
to attempt to reconstruct every other function. Now, theoretically 
any set of vectors can be chosen - however, when choosing 
basis vectors, two properties we find in 2 and 3 dimensional 
basis vectors are especially useful and can be generalized 
to our current case.

Firstly, the set of vectors $\hat{i}, \hat{j}, \hat{k}$ are all \textit{orthogonal} - 
meaning geometrically that any two different basis vectors are perpendicular.

Secondly, each of the basis vectors is \textit{normalized}, meaning 
in the 3-dimensional case that they all have a length of 1.

To rigorously define these properties in the case of function space,
the following definition for an Orthonormal Set is used:

% This is orthonormal

\begin{definition}[Orthonormal Set]
  A set of vectors $\mathit{U} = \left\{ \B{u}_1, ... , \B{u}_n \right\}$ is defined
  to be an orthonormal set iff
  \begin{equation}
    \braket{\B{u}_i| \B{u}_j} = \delta_{ij}, ~ \forall \B{u}_i, \B{u}_j \in  \mathit{U}
  \end{equation}
  With the Kronecker Delta $\delta_{ij}$ being defined as 
  \begin{equation}
    \delta_{ij} = 
    \begin{cases}
      0,  & i \neq j \\
      1, & i = j \\
    \end{cases}
  \end{equation}
  \label{def:OrthagonalSet}
\end{definition}

This provides a meaning to an orthogonal set of basis functions within function space.
For some set of functions $\mathit{U} = {\B{u}_n(x)}$ to be orthogonal,
combining the inner product from equation (\refeq{eq:standardIP}) with definition (\ref{def:OrthagonalSet})
gives:

\begin{equation}
  (\B{u}_i(x), \B{u}_j(x)) = \int_a^b \B{u}_i(x) \B{u}_j(x) dx  = \delta_{ij}
\end{equation}

\subsection{Defining The Closest Representation in an Orthonormal Basis}

Given a set of orthonormal basis vectors ${\B{u}_n(x)}$, my next goal was 
figuring out the process by which to determine coefficients ${c_n}$ such 
that the linear combination $\sum_{k=1}^n c_k \B{u}_k(x)$ either equaled
or best approximated the original arbitrary function $f(x)$. Speaking
in terms of a traditional 3 dimensional vector space with a vector $\vec{v}$, this 
amounts to determining its x, y and z coordinates, allowing the original 
vector to be written as a linear combination of the basis vectors $\hat{i}, \hat{j}, \hat{k}$.

As established earlier, establishing a vector representation in terms of a set of 
basis vectors is tremendously useful, because many important operations possessing the 
property of linearity can first be computed individually on the set of basis vectors,
then rescaled by the given coefficients to determine the effect of applying the function 
on the original vector itself. 

However, unlike vector spaces of finite dimension, the uncountable amount of ways
in which any two functions can differ makes it quite likely that no set of orthonormal vectors 
$\{\B{u}_n\}$ can span our entire vector space. Additionally, even in the case that there does exist a valid set of 
basis vectors, it is reasonable to believe that such a set would have an infinite number of 
elements. Because of this, it is immensely useful to construct a measure of how 
\textit{close} a given linear approximation is to the original function - giving a formal meaning to 
our intuitive sense of 'bad', 'good' or 'best' approximation. 

Let $\{\B{u}_n\}$ be an orthonormal set, and the $f(x)$ be any arbitrary function in our vector space. 
For a given set of scalars ${c_n}$, define the function 
\begin{equation}
  F(x) = c_1 \B{u}_1(x) + c_2 \B{u}_2(x) ... + c_n \B{u}_n(x)
  \label{eq:approxFn}
\end{equation}
to be the representation of $f(x)$ in the basis set. 
The difference between the representation and original function is another function of $x$
given by $\delta(x) = f(x) - F(x)$. Because $\delta(x)$ is defined as the linear combination
of $f(x)$, an element of $\funcSpace$, and $\basisSet$, all of which are elements of $\funcSpace$,
$\delta(x)$ must then also be a member of $\funcSpace$ - which in turn means that it has a valid 
inner product. 

Again using intuition from geometry, classically the `distance' between two vectors is 
the norm of the difference between those vectors - written in terms of the inner product as 
$\norm{\vec{v} - \vec{u}}^2 = (\vec{v} - \vec{u}, \vec{v} - \vec{u})$. This can be applied to 
say that the measure of how `close' an approximation function is from the original is given by 
\begin{equation}
  \norm{\delta(x)}^2 = \braket{f(x) - F(x) ~| ~f(x) - F(x)}
  \label{eq:distEq}
\end{equation}

Therefore, given the orthonormal basis $\basisSet$, the best approximation is defined 
to be the set of coefficients $\{c_n\}$ that minimize `distance' from equation (\refeq{eq:distEq}).

\subsection{Determining The Linear Representation of a Given Function}

Finally, to answer my research question, the last part of this essay will be dedicated
to determining how these coefficients relate to the given set of orthonormal basis functions, 
and attempting to find (if it exists) a basis such that the best linear approximation can reproduce exactly
any arbitrary function in the entire space.

Firstly, for a given $f(x)$ and basis $\basisSet$, define the set $\{c_n\}$ and corresponding function $F_c(x) = \sum_{k=1}^n c_k \B{u}_k$
such that, for any other linear combination $\{d_n\}$,
\begin{equation}
  \braket{f- F_c ~| ~ f - F_c} \leq \braket{f - F_d ~ | ~ f - F_d}
  \label{eq:bestApprox}
\end{equation}
with the two sides being equal if and only if the set $\{d_n\}$ is exactly equal to $\{c_n\}$.
(Note: the (x)'s have been dropped from here forwards for convenience, but are implied whenever not explicitly written).
Following from equation (\refeq{eq:bestApprox}) using the properties of 
linearity,
\begin{align}
  \braket{f- F_c ~| ~ f - F_c} &\leq \braket{f - F_d ~ | ~ f - F_d} \nonumber\\
  \braket{f|f} - 2\braket{f|F_c} + \braket{F_c|F_c} &\leq \braket{f|f} - 2\braket{f|F_d} + \braket{F_d|F_d} \nonumber\\
  - 2\braket{f|F_c} + \braket{F_c|F_c} &\leq - 2\braket{f|F_d} + \braket{F_d|F_d} 
  \label{eq:8}
\end{align}

Expanding $\braket{F_c|F_c}$ using $\B{u}_i(x) \B{u}_j(x) = 0 \leftrightarrow i \neq j$ and $\braket{\B{u}_i(x) |\B{u}_i(x)} = 1$, we get:

\begin{align}
  \braket{F_c | F_c} &= \int F_c(x) F_c(x) dx \nonumber\\
  &= \int(\sum_k^n c_k \B{u}_k(x) \sum_k^n c_k \B{u}_k(x)) dx \nonumber\\
  &= \int \sum_k^n (c_k \B{u}_k(x)) ^2 dx \nonumber\\
  &=  \sum_k^n  {c_k}^2 \int  (\B{u}_k(x)) ^2 dx \nonumber\\
  &=  \sum_k^n {c_k}^2  \braket{\B{u}_k | \B{u}_k } \nonumber\\
  &=  \sum_k^n {c_k}^2 \label{eq:9}
\end{align}


Similarly expanding $\braket{f|F_c}$, we get:
\begin{align}
  \braket{f| F_c} &= \int f(x) F_c(x) dx \nonumber\\
  &= \int \sum_k^n c_k (f(x) \B{u}_k(x)) dx \nonumber\\
  &=  \sum_k^n c_k \braket{f(x)| \B{u}_k }  \label{eq:10}
\end{align}

And plugging equations (\refeq{eq:9}) and (\refeq{eq:10}) into (\refeq{eq:8}) to get:
\begin{align}
    2\braket{f|F_c} - \braket{F_c|F_c} &\geq 2\braket{f|F_d} - \braket{F_d|F_d}   \nonumber\\
    2 \sum_k^n c_k \braket{f(x)| \B{u}_k } - \sum_k^n {c_k}^2 &\geq 2 \sum_k^n d_k \braket{f(x)| \B{u}_k } - \sum_k^n {d_k}^2 \nonumber\\
    \sum_k^n(2 c_k \braket{f(x)| \B{u}_k } - {c_k}^2) &\geq \sum_k^n(2 d_k \braket{f(x)| \B{u}_k } - {d_k}^2)  \label{eq:almost}
\end{align}
Because equation (\refeq{eq:almost}) is true for any set $\{d_k\}$, even sets that only differ 
in one single coefficient, equation (\refeq{eq:almost}) must be true term for term - that is to say:
\begin{align}
  (2 c_k \braket{f(x)| \B{u}_k } - {c_k}^2) &\geq (2 d_k \braket{f(x)| \B{u}_k } - {d_k}^2) ~ \forall k \in n, \\
  (2 c_k \braket{f(x)| \B{u}_k } - {c_k}^2) - \braket{f(x)| \B{u}_k }^2  &\geq (2 d_k \braket{f(x)| \B{u}_k } - {d_k}^2) - \braket{f(x)| \B{u}_k }^2 \nonumber\\
  -({c_k} - \braket{f(x)| \B{u}_k })^2  &\geq -({d_k} - \braket{f(x)| \B{u}_k })^2 \nonumber\\
  ({c_k} - \braket{f(x)| \B{u}_k })^2  &\leq ({d_k} - \braket{f(x)| \B{u}_k })^2 ~ \forall d_k \in \R \nonumber\\
  \therefore ~ c_k &= \braket{f(x)| \B{u}_k }, ~ \forall k \in n \label{eq:bestApproxEqn}
\end{align}

Therefore, from equation (\refeq{eq:bestApproxEqn}), the representation of a function in an orthonormal
basis is obtained by taking the inner product between that function and each basis function.

% \begin{definition}
%   Given two vector spaces $\mathbb{V}, \mathbb{W}$ defined over the same scalar field $\mathbb{S}$,
%   a function $f(\vec{v}) : \mathbb{V} \to \mathbb{W}$ defined over the entire vector field
%   $\mathbb{V}$ is said to be linear iff it satisfies the following conditions:

%   \begin{enumerate}
%     \item Scalar Associativity: $f(c \vec{v}) = c f(\vec{v}) ~ \forall \vec{v} \in \mathbb{V}, c \in \mathbb{S} $
%     \item Distributive Across Vector Addition: $f(\vec{v}_1 + \vec{v}_2) = f(\vec{v}_1) + f(\vec{v}_2) ~ \forall \vec{v}_1, \vec{v}_2 \in \mathbb{V}$
%   \end{enumerate}
%   \label{def:linearity}
% \end{definition}

% \begin{definition}
%   A Linear Operator is defined to be a function from one vector space to itself $f(\vec{v}) : \mathbb{V} \to \mathbb{V}$ that satisfies
%   the properties of linearity outlined in definition (\ref{def:linearity}).
%   \label{def:linearOperator}
% \end{definition}


% \begin{definition}
%   Let $\B{x} \in \mathbb{V}$ for some vector space $\mathbb{V}$, and let $\xi \in \mathscr{L}(\mathbb{V}, \mathbb{V})$ be defined
%   as a linear operator over $\mathbb{V}$. 
%   For some scalar $\lambda_r$, $\B{x}_r$ is defined to be an eigenvector of the eigenvalue $\lambda_r$ iff
%   \begin{equation}
%     \xi \B{x}_r = \lambda_r \B{x}_r
%   \end{equation}
%   \label{def:Eigenvalues}
% \end{definition}





% \begin{theorem}
%   Let $\mathscr{F}_C(\mathcal{D})$ be defined as the subset of $\mathscr{F}(\mathcal{D})$ containing all functions
%   which are infinitely differentiable, or for which $\frac{d^n}{(dx)^n} f(x)$ exists 
%   for $n \in \mathbb{N}$ over the domain $\mathcal{D}$. The derivative then
%   defines a linear operator over the vector space $\mathscr{F}_C(\mathcal{D})$.
% \end{theorem}

% \begin{proof}
%   A linear operator as defined in (\ref{def:linearOperator}) refers to a mapping of a vector space into itself that satisfies the 
%   conditions of linearity. 

%   First, showing that the derivate maps $ \mathscr{F}_C(\mathcal{D})$ into itself:

%   Take any infinitely differentiable function $f(x)$ over $\mathcal{D}$. 
%   Then for any $n \in \mathbb{N}$, $\frac{d^{n+1}}{(dx)^{n+1}} f(x)$ is also defined.
%   $\frac{d^{n+1}}{(dx)^{n+1}} f(x) = \frac{d^n}{(dx)^n} \frac{d}{dx} f(x)$, hence the derivative
%   of any infinitely differentiable function is also infinitely differentiable, and so 
%   if $f(x) \in \mathscr{F}_C(\mathcal{D})$, then $\frac{d}{dx} f(x) \in  \mathscr{F}_C(\mathcal{D})$.

%   Next, showing that the derivative satisfies the conditions of linearity:
%   \begin{enumerate}
%     \item Scalar Associativity: $\frac{d}{dx} (c f(x)) \triangleq c \frac{d}{dx} f(x)$
%     \item Distributivity: $\frac{d}{dx}(f(x) + g(x)) \triangleq \frac{d}{dx} f(x) + \frac{d}{dx} g(x)$
%   \end{enumerate}

%   Therefore the derivative satisfies a linear operator over $\mathscr{F}_C(\mathcal{D})$.
% \end{proof}


% \subsection{Constructing An Inner Product On $\mathscr{F}_C(\mathcal{D})$}
% \begin{theorem}

% \end{theorem}


% \begin{theorem}
%   Let $\mathbb{P}^n$ be the set of all real n-degree polynomials of the form
%   \begin{equation}
%     x^n + a_1 x^{n-1} + ... + a_{n-1} x^1 + a_n = 0
%   \end{equation}
%   where the coefficients $a_i$, besides the condition that they 
%   are all real, can have any values whatsoever.

%   This set is one-to-one isomorphic to the set of all root-forms of degreen $n$,
%   written as 

%   \begin{equation}
%     (x-b_1) (x-b_2) ... (x-b_{n-1}) (x-b_{n}) = 0, 
%     \label{eq:factorForm}
%   \end{equation}
%   with $b_i \in \mathbb{C}$, and the condition that,
%   for each $i \in n$, there also exists a unique $j \in n$ such that
%   $b_i = \overline{b_j}$, with the two indices $i, j$ possibly being equal
%   in the case where $b_i$ is real.

% \end{theorem}

% First this will be proven in the case of $n = 2$.
% \begin{proof}
%   Firstly, note that the set $\mathbb{P}^n$ is isomorphic to 
%   the set $\mathbb{R}^n$. The coefficients can be any real number 
%   independent of the others, and each unique set of real coefficients 
%   gives a unique polynomial not expressible in any other way. 

%   This means that $\mathbb{P}^2$ is isomorphic to $\mathbb{R}^2$.
%   Next, the factor form in equation (\refeq{eq:factorForm}) for the 
%   case of $n = 2$ can be divided into two cases: either both $b_1, b_2$ are real,
%   or $b_1 = \overline{b_2}$. This accounts exhaustively for all of the 
%   factor form cases of n = 2 with the conditions outlined above.
  
%   First consider the case where both coefficients are real. 
  
  
%   This 
%   is isomorphic to the set 
% \end{proof}




% \begin{theorem}
%   Let $f(x) : \R \to \R$ be well defined and definite increasing 
%   in interval $(a, b) \in \R$. Then $f(x)$ must be continuos on an infinite 
%   set of points $C \subset (a , b)$ within the interval.
% \end{theorem}


% \begin{proof}
%   Let $y_a = f(a)$. 
%   By definition of continuity, there exists some $\epsilon_a$ such that 
%   no matter how small $\delta_a > 0$ is chosen, $\abs{f(a + \delta_a)- f(a)} < \epsilon_a$.
  
%   Given that $f(x)$ is definite increasing within the interval, $f(a + \delta_a)- f(a) < \epsilon_a$.

% \end{proof}



% \begin{theorem}
%   Let $\mathscr{F}$ denote the set of $f(\B{x}, t) \to \R^+$ with unity integral.
%   $C(\B{p}_1, \B{p}_2, \vec{R}) \to \B{p}^\prime$ is the collision kernel.
%   To find how $f(\B{x}, t)$ changes with time, you can analyze for each point, what is the 
%   probability that it maps somewhere else taking that as losses. And then look at the probability that
%   something maps to it. 
% \end{theorem}

% \end{proof}
\pagebreak
\medskip

\addcontentsline{toc}{section}{References}
\bibliographystyle{apalike}
\bibliography{references}

\end{document}